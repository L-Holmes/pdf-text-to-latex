\documentclass[12pt]{article}

\usepackage[T1]{fontenc}\usepackage{graphicx}\graphicspath{ {./images/} } \usepackage{float}\begin{document}

(Q)
Describe: ...
\clearpage
SCC361: Artificial Intelligence\\
Week 1: Introduction to Artificial Intelligence and Machine Learning\\
Dr Bryan M. Williams\\
School of Computing and Communications, Lancaster University\\
Office: InfoLab21 C40 Email: b.williams6@lancaster.ac.uk\\
1\\
\begin{figure}[H]
\includegraphics[width=0.5\linewidth]{page1-image-1.png}
\end{figure}
\clearpage
(Q)
Describe: 2
\clearpage
SCC361: Artificial Intelligence\\
\section{2}
Lectures, Materials and Expectations\\
Introduction to the Module\\
Overview of Artificial Intelligence\\
Overview of Machine Learning\\
\begin{figure}[H]
\includegraphics[width=0.5\linewidth]{page1-image-1.png}
\end{figure}
\clearpage
(Q)
Describe: • This version is unedited
\clearpage
Playing this Video\\
\section{• This version is unedited}
\begin{itemize}
  \item In general, it might be slow for some people
  \item Vary the playback speed to suit you preferred pace
  \item In live sessions, you can ask questions at any stage, but the 
Questions? slides will give you a specific opportunity to ask \\
questions\\
  \item While watching, use the Questions? slides as stop points for 
coffee breaks, notes etc\\
\end{itemize}
3\\
\begin{figure}[H]
\includegraphics[width=0.5\linewidth]{page1-image-1.png}
\end{figure}
\clearpage
(Q)
Describe: • All our content is expected to meet the UK 
\clearpage
Accessibility\\
\section{• All our content is expected to meet the UK }
accessibility requirements\\
\end{itemize}
  \item We have done our best to ensure that that is the 
case with these course materials\\
  \item However, if any course material or part of its content 
is inaccessible in anyway to any individual or group, \\
kindly let us know.\\
\end{itemize}
4\\
\begin{figure}[H]
\includegraphics[width=0.5\linewidth]{page1-image-1.png}
\end{figure}
\clearpage
(Q)
Describe: Attendance Check-in
\clearpage
Be sure to check in to all timetabled sessions using \\
\section{Attendance Check-in}
To check in:\\
\begin{itemize}
  \item Check the Attendance Hub in iLancaster
  \item Click Check In
  \item Wait for the “You are checked in” confirmation page
  \item Here is a the demo
Please DO NOT leave a timetabled session without your\\
attendance being registered\\
Attendance Check-in\\
\end{itemize}
5\\
\begin{figure}[H]
\includegraphics[width=0.5\linewidth]{page1-image-1.png}
\end{figure}
\clearpage
(Q)
Describe: Before the session:
\clearpage
Online Sessions on Teams\\
\section{Before the session:}
\end{itemize}
  \item Follow the Moodle link to join the lecture
  \item Ensure that your speakers, headsets are connected and working
\end{itemize}
6\\
\begin{figure}[H]
\includegraphics[width=0.5\linewidth]{page1-image-1.png}
\end{figure}
\clearpage
(Q)
Describe: During the lectures:
\clearpage
Online Sessions on Teams\\
\section{During the lectures:}
\begin{itemize}
  \item Turn your webcam off
  \item Turn your microphone off
\end{itemize}
7\\
\begin{figure}[H]
\includegraphics[width=0.5\linewidth]{page1-image-1.png}
\end{figure}
\clearpage
(Q)
Describe: During the lectures:
\clearpage
Online Sessions on Teams\\
\section{During the lectures:}
\end{itemize}
  \item Use  chat  appropriately. Not closely monitored during lectures.
For live Q\&A sessions:\\
  \item Raise your hand to ask questions. Lower it afterwards.
  \item When called, turn on your mic (and cam if you wish). Remember to turn them off 
afterwards.\\
Post additional questions on the SCC361 Moodle Forum\\
\end{itemize}
8\\
\begin{figure}[H]
\includegraphics[width=0.5\linewidth]{page1-image-1.png}
\end{figure}
\clearpage
(Q)
Describe: After the lectures:
\clearpage
Online Sessions on Teams\\
\section{After the lectures:}
\begin{itemize}
  \item The recorded content of the live sessions will be made available after the session on the 
Moodle Space\\
\end{itemize}
9\\
\begin{figure}[H]
\includegraphics[width=0.5\linewidth]{page1-image-1.png}
\end{figure}
\clearpage
(Q)
Describe: If you are struggling to watch the videos on Moodle:
\clearpage
The lectures can be watched on the Moodle space.\\
\section{If you are struggling to watch the videos on Moodle:}
\end{itemize}
  \item Download the video and caption file (*.vtt) from Moodle
  \item Download the free, open source VLC Media player: 
https://www.videolan.org/vlc/index.en-GB.html\\
  \item Open video file in VLC and add caption file
Note:\\
All learning materials: slides, videos and caption\\
files are @Lancaster University copyright and are \\
not to be shared or distributed.\\
Using Materials Offline\\
\end{itemize}
10\\
\begin{figure}[H]
\includegraphics[width=0.5\linewidth]{page1-image-1.png}
\end{figure}
\clearpage
(Q)
Describe: • Colluding with a classmate or someone else to do your work
\clearpage
Passing off someone else’s work as your own, including:\\
\section{• Colluding with a classmate or someone else to do your work}
\begin{itemize}
  \item Submitting code written by someone else
  \item Paying for someone else to do your work
  \item Adapting code by someone else with only a minor modification
  \item Course work is submitted on Moodle and will be checked automatically for plagiarism!
Plagiarism\\
\end{itemize}
11\\
\end{itemize}
  \item \begin{figure}[H]
\includegraphics[width=0.5\linewidth]{page1-image-1.png}
\end{figure}
\clearpage
(Q)
Describe: • The use of these is governed by existing policies that you are all currently bound by and 
\clearpage
Online tools will be used to facilitate some aspects of learning e.g. Moodle, Teams, etc.
\section{• The use of these is governed by existing policies that you are all currently bound by and }
have agreed to\\
  \item Academic malpractice and plagiarism still applies online
  \item Direct sharing of code, sharing solutions and/or partial solutions with other students, 
either privately or in an open chat, is not acceptable\\
Online Learning Expectations\\
\end{itemize}
12\\
\begin{itemize}
  \item \begin{figure}[H]
\includegraphics[width=0.5\linewidth]{page1-image-1.png}
\end{figure}
\clearpage
(Q)
Describe: the Internet
\clearpage
Don’t forget, these are your fellow students and staff, not some anonymous person on 
\section{the Internet}
  \item If you’re not sure if you should post or share something, please ask first
  \item If you see content or a post that you don’t like, in the first instance, message or email 
the course tutor to alert them to it\\
  \item We want these tools to be used; they will give you the best online experience!
  \item However, we are asking that you use them sensibly and with respect
Online Learning Expectations\\
\end{itemize}
13\\
\begin{figure}[H]
\includegraphics[width=0.5\linewidth]{page1-image-1.png}
\end{figure}
\clearpage
(Q)
Describe: • Lectures, labs etc., be punctual
\clearpage
Attendance:\\
\section{• Lectures, labs etc., be punctual}
Active learning:\\
\end{itemize}
  \item Read around (explore) the subject
  \item Use recommended books and available online resources
  \item Ask questions, try things yourself, keep notes
  \item Have a study plan, get a study partner
Integrity:\\
  \item Honesty, no plagiarism/ result manipulation
What do we expect from you?\\
\end{itemize}
14\\
\begin{itemize}
  \item \begin{figure}[H]
\includegraphics[width=0.5\linewidth]{page1-image-1.png}
\end{figure}
\clearpage
(Q)
Describe: • Provide references to follow up
\clearpage
Lecture slides and videos will be available on Moodle
\section{• Provide references to follow up}
  \item TAs will be available to ensure that the labs are running smoothly
  \item Arrange extra support if you are struggling and let us know on time
  \item Provide prompt feedback on formative coursework
  \item In extreme cases, respond to coursework questions outside the labs
  \item We encourage you to maximise the use of lab sessions for all coursework related 
questions\\
How can we help?\\
\end{itemize}
15\\
\end{itemize}
  \item \begin{figure}[H]
\includegraphics[width=0.5\linewidth]{page1-image-1.png}
\end{figure}
\clearpage
(Q)
Describe: • Use the course forum on Moodle
\clearpage
Use the labs to ask TAs/Tutors for help
\section{• Use the course forum on Moodle}
  \item Check other available (online) resources
  \item Drop me an email/ ask on Teams chat
  \item There might be delays in replying
How to get help\\
\end{itemize}
16\\
\begin{figure}[H]
\includegraphics[width=0.5\linewidth]{page1-image-1.png}
\end{figure}
\clearpage
(Q)
Describe: 17
\clearpage
Questions?\\
\section{17}
\begin{figure}[H]
\includegraphics[width=0.5\linewidth]{page1-image-1.png}
\end{figure}
\clearpage
(Q)
Describe: 18
\clearpage
SCC361: Artificial Intelligence\\
\section{18}
Lectures, Materials and Expectations\\
Introduction to the Module\\
Overview of Artificial Intelligence\\
Overview of Machine Learning\\
\begin{figure}[H]
\includegraphics[width=0.5\linewidth]{page1-image-1.png}
\end{figure}
\clearpage
(Q)
Describe: • Expected Learning Outcomes
\clearpage
In this video\\
\section{• Expected Learning Outcomes}
\begin{itemize}
  \item Teaching Staff
  \item Lecture Plan
  \item Teaching Structure
  \item Assessment
\end{itemize}
19\\
\begin{figure}[H]
\includegraphics[width=0.5\linewidth]{page1-image-1.png}
\end{figure}
\clearpage
(Q)
Describe: We will to be able to:
\clearpage
Expected Learning Outcomes\\
\section{We will to be able to:}
\end{itemize}
  \item understand AI concepts, applications and trends
  \item understand machine learning terms
  \item train machine learning models for specific tasks
  \item learn implement simple AI-based systems
  \item learn how to evaluate the performance of AI systems
\end{itemize}
20\\
\begin{figure}[H]
\includegraphics[width=0.5\linewidth]{page1-image-1.png}
\end{figure}
\clearpage
(Q)
Describe: 21
\clearpage
Teaching Staff\\
\section{21}
Dr Bryan M. Williams\\
Weeks 1-5\\
Dr Hossein Rahmani\\
Weeks 6-10\\
Module Convenor\\
Teaching Assistants Group 1 Group 2 Group 3 Group 4 Group 5\\
Mona Alghamdi\\
Somayeh Bazin\\
Piotr Daniszewski\\
Oishi Deb\\
Ovini Gunasekera\\
Yuri Tavares dos Passos\\
\begin{figure}[H]
\includegraphics[width=0.5\linewidth]{page1-image-1.png}
\end{figure}
\clearpage
(Q)
Describe: Weeks 1-5
\clearpage
Lecture Plan\\
\section{Weeks 1-5}
1. Introduction to Artificial Intelligence \\
and Machine Learning\\
2. Features in Machine Learning and \\
Feature Extraction\\
3. Computer Vision and Natura Language \\
Processing\\
4. Clustering and Classification\\
5. Artificial Neural Networks\\
Weeks 6-10:\\
6. Genetic Algorithms\\
7. Naïve Bayesian Classifier\\
8. Decision Tree Classifier\\
9. Introduction to Deep Neural Networks\\
10. Introduction to Convolutional Neural \\
Networks\\
22\\
\begin{figure}[H]
\includegraphics[width=0.5\linewidth]{page1-image-1.png}
\end{figure}
\clearpage
(Q)
Describe: Lectures:
\clearpage
Teaching Structure\\
\section{Lectures:}
\begin{itemize}
  \item Weeks 1-10
  \item Online only
  \item Mondays: 14.00-15.00
  \item Tuesdays: 17.00-18.00
Labs:\\
  \item Weeks 1-10
  \item Blended: in-person and online
  \item Wednesdays: 11.00-13.00
  \item Thursdays: 10.00-12.00
  \item Thursdays: 16.00-18.00
  \item Fridays: 11.00-13.00
  \item Fridays: 16.00-18.00
\end{itemize}
23\\
\begin{figure}[H]
\includegraphics[width=0.5\linewidth]{page1-image-1.png}
\end{figure}
\clearpage
(Q)
Describe: Labs:
\clearpage
Teaching Structure\\
\section{Labs:}
24\\
Group Day Time Room\\
SCC361/P01/01 Wednesday 11:00-13:00 FST B076\\
SCC361/P01/02 Thursday 16:00-18:00 FST B076\\
SCC361/P01/03 Friday 16:00-18:00 (Weeks 1-4, 6, 8, 10)\\
14:00-16:00 (Weeks 5, 7, 9)\\
FST B070 (Weeks 1-4)\\
FST B074 (Weeks 5, 7, 9)\\
FST B080 (Weeks 6, 8, 10)\\
SCC361/P01/04 Thursday 10:00-12:00 FST B070\\
SCC361/P01/05 Friday 11:00-13:00 FST B080\\
\begin{figure}[H]
\includegraphics[width=0.5\linewidth]{page1-image-1.png}
\end{figure}
\clearpage
(Q)
Describe: • CW1 (20marks):
\clearpage
2 Courseworks: 40\%\\
\section{• CW1 (20marks):}
\end{itemize}
  \item Submission: On Moodle
  \item Deadline: 5pm Friday 12th November, 2021
  \item CW2 (20marks):
  \item Details: to be confirmed
Exam: 60\%\\
  \item Next semester in 2021
  \item Date: To be confirmed
Assessment\\
\end{itemize}
25\\
\begin{figure}[H]
\includegraphics[width=0.5\linewidth]{page1-image-1.png}
\end{figure}
\clearpage
(Q)
Describe: 26
\clearpage
Questions?\\
\section{26}
\begin{figure}[H]
\includegraphics[width=0.5\linewidth]{page1-image-1.png}
\end{figure}
\clearpage
(Q)
Describe: 27
\clearpage
SCC361: Artificial Intelligence\\
\section{27}
Lectures, Materials and Expectations\\
Introduction to the Module\\
Overview of Artificial Intelligence\\
Overview of Machine Learning\\
\begin{figure}[H]
\includegraphics[width=0.5\linewidth]{page1-image-1.png}
\end{figure}
\clearpage
(Q)
Describe: • Wk1: Intro to Artificial Intelligence \& Machine Learning
\clearpage
Part 1 (Weeks 1 – 5):\\
\section{• Wk1: Intro to Artificial Intelligence \& Machine Learning}
\begin{itemize}
  \item Wk2: What are features and how to extract features from texts, 
images, etc.\\
  \item Wk3: Computer Vision and Natural Language Processing
  \item Wk4: Clustering \& Classification
  \item Wk5: Intro to Artificial Neural Networks \& Review of Previous 
Lectures\\
Part 2 (Weeks 6 – 10):\\
  \item Dr Hossein Rahmani (GAs, NBCs, DTCs, DNNs, CNNs)
Welcome to SCC361\\
\end{itemize}
28\\
\begin{figure}[H]
\includegraphics[width=0.5\linewidth]{page1-image-1.png}
\end{figure}
\clearpage
(Q)
Describe: • Application, history, foundations of AI
\clearpage
Artificial Intelligence: An Overview\\
\section{• Application, history, foundations of AI}
\end{itemize}
  \item Definition of AI
  \item Goals of AI
  \item AI and the Society
  \item Benefits
  \item Risk and Challenges
  \item Ethical Issues
This Week’s Lectures\\
\end{itemize}
29\\
\begin{figure}[H]
\includegraphics[width=0.5\linewidth]{page1-image-1.png}
\end{figure}
\clearpage
(Q)
Describe: • AI and ML, Definitions of ML, How to learn
\clearpage
Overview of Machine Learning\\
\section{• AI and ML, Definitions of ML, How to learn}
Types of Machine Learning\\
\begin{itemize}
  \item Supervised, unsupervised, semi supervised
Supervised Learning\\
  \item Classification and regression
Unsupervised Learning\\
  \item Clustering and association
This Week’s Lectures\\
\end{itemize}
30\\
\end{itemize}
  \item \begin{figure}[H]
\includegraphics[width=0.5\linewidth]{page1-image-1.png}
\end{figure}
\clearpage
(Q)
Describe: Agents 2ed. Poole \& Mackworth 2017
\clearpage
Artificial Intelligence: Foundations of Computational 
\section{Agents 2ed. Poole \& Mackworth 2017}
  \item Artificial Intelligence A Modern Approach, Russell \& 
Norvig, 2016 (chapters 1,2,5,6)\\
  \item Deep Learning, Goodfellow et al. , 2016
  \item Deep Learning with PyTorch, Stevens et al, 2020.
  \item Machine Learning, T. M. Mitchell, 1997
  \item Artificial Intelligence on Wikipedia
  \item Many online resources
Recommended Reading\\
\end{itemize}
31\\
\begin{figure}[H]
\includegraphics[width=0.5\linewidth]{page1-image-1.png}
\end{figure}
\clearpage
(Q)
Describe: 32
\clearpage
SCC361: Week 1\\
\section{32}
\begin{itemize}
  \item AI Overview
  \item Definition of AI
  \item Goals of AI
  \item AI and Society
Artificial Intelligence\\
  \item Machine Learning Overview
  \item Types of Machine Learning
  \item Supervised Learning
  \item Unsupervised Learning
Introduction to Machine Learning\\
\end{itemize}
\begin{figure}[H]
\includegraphics[width=0.5\linewidth]{page1-image-1.png}
\end{figure}
\clearpage
(Q)
Describe: 33
\clearpage
AI in Real Life\\
\section{33}
\begin{figure}[H]
\includegraphics[width=0.5\linewidth]{page1-image-1.png}
\end{figure}
\clearpage
(Q)
Describe: 34
\clearpage
AI in Science Fiction Movies\\
\section{34}
\begin{figure}[H]
\includegraphics[width=0.5\linewidth]{page1-image-1.png}
\end{figure}
\clearpage
(Q)
Describe: 35
\clearpage
AI in Science Fiction Movies\\
\section{35}
\begin{figure}[H]
\includegraphics[width=0.5\linewidth]{page1-image-1.png}
\end{figure}
\clearpage
(Q)
Describe: 36
\clearpage
AI in Music\\
\section{36}
\begin{figure}[H]
\includegraphics[width=0.5\linewidth]{page1-image-1.png}
\end{figure}
\clearpage
(Q)
Describe: 37
\clearpage
AI in Agriculture\\
\section{37}
\begin{figure}[H]
\includegraphics[width=0.5\linewidth]{page1-image-1.png}
\end{figure}
\clearpage
(Q)
Describe: 38
\clearpage
AI in Delivery Services\\
\section{38}
\begin{figure}[H]
\includegraphics[width=0.5\linewidth]{page1-image-1.png}
\end{figure}
\clearpage
(Q)
Describe: 39
\clearpage
AI in Self-Driving Vehicles\\
\section{39}
\begin{figure}[H]
\includegraphics[width=0.5\linewidth]{page1-image-1.png}
\end{figure}
\clearpage
(Q)
Describe: 40
\clearpage
AI in Medicine\\
\section{40}
\begin{figure}[H]
\includegraphics[width=0.5\linewidth]{page1-image-1.png}
\end{figure}
\clearpage
(Q)
Describe: 41
\clearpage
AI in Medicine\\
\section{41}
\begin{figure}[H]
\includegraphics[width=0.5\linewidth]{page1-image-1.png}
\end{figure}
\clearpage
(Q)
Describe: 42
\clearpage
AI in Medicine\\
\section{42}
\begin{figure}[H]
\includegraphics[width=0.5\linewidth]{page1-image-1.png}
\end{figure}
\clearpage
(Q)
Describe: 43
\clearpage
AI in Medicine\\
\section{43}
\begin{figure}[H]
\includegraphics[width=0.5\linewidth]{page1-image-1.png}
\end{figure}
\clearpage
(Q)
Describe: 44
\clearpage
AI in Security\\
\section{44}
\begin{figure}[H]
\includegraphics[width=0.5\linewidth]{page1-image-1.png}
\end{figure}
\clearpage
(Q)
Describe: 45
\clearpage
AI in Forensic Identification\\
\section{45}
https://h-unique.lancaster.ac.uk/\\
\end{itemize}
  \item \begin{figure}[H]
\includegraphics[width=0.5\linewidth]{page1-image-1.png}
\end{figure}
\clearpage
(Q)
Describe: • Text classification: sentiment, topic
\clearpage
Web search engines
\section{• Text classification: sentiment, topic}
  \item Spam filtering etc
  \item Machine translation
  \item Question answering
  \item Recommender Systems
AI in Natural Language Processing\\
\end{itemize}
46\\
\begin{figure}[H]
\includegraphics[width=0.5\linewidth]{page1-image-1.png}
\end{figure}
\clearpage
(Q)
Describe: • Siri, Alexa, Cortana, Google Assistant
\clearpage
Speech Technologies\\
\section{• Siri, Alexa, Cortana, Google Assistant}
\begin{itemize}
  \item Automatic Speech Recognition
  \item Dialogue systems
AI in Natural Language Processing\\
\end{itemize}
47\\
\begin{figure}[H]
\includegraphics[width=0.5\linewidth]{page1-image-1.png}
\end{figure}
\clearpage
(Q)
Describe: 48
\clearpage
Brief History of AI\\
\section{48}
1940-1950: Early Days\\
\end{itemize}
  \item 1943: McCulloch \& Pitts: 
Boolean Circuit Model of \\
Brain\\
  \item 1950: Turing's “Computing 
Machinery and \\
Intelligence”\\
1950-1970: Excitement\\
  \item 1950s: Early AI programs: 
Samuel's checkers \\
program, Newell \& \\
Simon's Logic Theorist, \\
Gelernter's Geometry \\
Engine\\
  \item 1956: Dartmouth meeting: 
“Artificial Intelligence” \\
adopted\\
  \item 1965: Robinson's complete 
algorithm for logical \\
reasoning\\
1970-1990: Knowledge-\\
based approaches\\
  \item 1969-79: Early 
development of \\
knowledge-based systems\\
  \item 1980-88: Expert systems
industry booms\\
  \item 1988-93: Expert systems 
industry busts: “AI Winter”\\
Lesson Notes from Nikita Kitaev, University of California, Berkeley\\
\end{itemize}
\begin{figure}[H]
\includegraphics[width=0.5\linewidth]{page1-image-1.png}
\end{figure}
\clearpage
(Q)
Describe: 49
\clearpage
Brief History of AI\\
\section{49}
1990 – 2012: \\
Statistical approaches \\
+ subfield  expertise:\\
\begin{itemize}
  \item Resurgence of probability, 
focus on uncertainty\\
  \item General increase in technical 
depth\\
  \item Agents and machine learning 
systems… “AI Spring”?\\
2012 – now: \\
Excitement:\\
  \item Big data, big compute, deep 
neural networks\\
  \item Some re-unification of 
subfields\\
  \item AI used in many industries
Lesson Notes from Nikita Kitaev, University of California, Berkeley\\
\end{itemize}
\begin{figure}[H]
\includegraphics[width=0.5\linewidth]{page1-image-1.png}
\end{figure}
\clearpage
(Q)
Describe: 50
\clearpage
Foundations of Artificial Intelligence\\
\section{50}
Philosophy\\
Mathematics\\
Economics\\
Neuroscience\\
Psychology\\
Computer \\
engineering\\
Control \\
theory and \\
cybernetics\\
\begin{figure}[H]
\includegraphics[width=0.5\linewidth]{page1-image-1.png}
\end{figure}
\clearpage
(Q)
Describe: 51
\clearpage
Questions?\\
\section{51}
\begin{figure}[H]
\includegraphics[width=0.5\linewidth]{page1-image-1.png}
\end{figure}
\clearpage
(Q)
Describe: 52
\clearpage
The Thinking Machine\\
\section{52}
\end{itemize}
  \item Can machines really think?
  \item Interviews by some of the AI pioneers in the 
1960s:\\
  \item Jerome Wiesner,
  \item Oliver Selfridge,
  \item Claude Shannon
  \item Can a robot marry my daughter?
  \item Can AI translate write poetry?
\end{itemize}
\begin{figure}[H]
\includegraphics[width=0.5\linewidth]{page1-image-1.png}
\end{figure}
\clearpage
(Q)
Describe: 53
\clearpage
Human Intelligence\\
\section{53}
\begin{figure}[H]
\includegraphics[width=0.5\linewidth]{page1-image-1.png}
\end{figure}
\clearpage
(Q)
Describe: • “The exciting new effort to make  
\clearpage
Approach 1: Thinking Humanly\\
\section{• “The exciting new effort to make  }
computers think ...machines with  minds, \\
in the full and literal  sense.” (Haugeland, \\
1985)\\
\begin{itemize}
  \item “[The automation of] activities  that we 
associate with human  thinking, activities \\
such as  decision-making, problem \\
solving,  learning ...” (Bellman, 1978)\\
What is Artificial Intelligence?\\
\end{itemize}
54\\
Artificial Intelligence: A Modern Approach, 2016, Russell \& Norvig\\
\begin{figure}[H]
\includegraphics[width=0.5\linewidth]{page1-image-1.png}
\end{figure}
\clearpage
(Q)
Describe: • “The art of creating machines that  
\clearpage
Approach 2: Acting Humanly\\
\section{• “The art of creating machines that  }
perform functions that require  \\
intelligence when performed by  people.” \\
(Kurzweil, 1990)\\
\end{itemize}
  \item “The study of how to make  computers do 
things at which, at  the moment, people \\
are better.”  (Rich and Knight, 1991)\\
What is Artificial Intelligence?\\
\end{itemize}
55\\
Artificial Intelligence: A Modern Approach, 2016, Russell \& Norvig\\
\begin{figure}[H]
\includegraphics[width=0.5\linewidth]{page1-image-1.png}
\end{figure}
\clearpage
(Q)
Describe: • “The study of mental faculties  through 
\clearpage
Approach 3: Thinking Rationally\\
\section{• “The study of mental faculties  through }
the use of computational  models.” \\
(Charniak and  McDermott, 1985)\\
\begin{itemize}
  \item “The study of the computations  that 
make it possible to perceive,  reason, and \\
act.” (Winston, 1992)\\
What is Artificial Intelligence?\\
\end{itemize}
56\\
Artificial Intelligence: A Modern Approach, 2016, Russell \& Norvig\\
\begin{figure}[H]
\includegraphics[width=0.5\linewidth]{page1-image-1.png}
\end{figure}
\clearpage
(Q)
Describe: • “Computational Intelligence is the  study 
\clearpage
Approach 4: Acting Rationally\\
\section{• “Computational Intelligence is the  study }
of the design of intelligent  agents.” \\
(Poole et al., 1998)\\
\end{itemize}
  \item “AI ... is concerned with intelligent  
behavior in artifacts.” (Nilsson, 1998)\\
What is Artificial Intelligence?\\
\end{itemize}
57\\
Artificial Intelligence: A Modern Approach, 2016, Russell \& Norvig\\
\begin{figure}[H]
\includegraphics[width=0.5\linewidth]{page1-image-1.png}
\end{figure}
\clearpage
(Q)
Describe: 58
\clearpage
Approaches to defining AI\\
\section{58}
Artificial Intelligence: A Modern Approach, 2016, Russell \& Norvig\\
Human Rational\\
Thinking Systems that think like humans Systems that think rationally\\
Acting Systems that act like humans Systems that act rationally\\
\begin{figure}[H]
\includegraphics[width=0.5\linewidth]{page1-image-1.png}
\end{figure}
\clearpage
(Q)
Describe: 59
\clearpage
Approaches to defining AI\\
\section{59}
Artificial Intelligence: A Modern Approach, 2016, Russell \& Norvig\\
Human Rational\\
Thinking\\
Systems that think like humans\\
\begin{itemize}
  \item Cognitive modelling approach
  \item Introspection, psychological  
experiments, brain imaging\\
  \item Cognitive Science
Systems that think rationally\\
Acting Systems that act like humans Systems that act rationally\\
\end{itemize}
\begin{figure}[H]
\includegraphics[width=0.5\linewidth]{page1-image-1.png}
\end{figure}
\clearpage
(Q)
Describe: 60
\clearpage
Approaches to defining AI\\
\section{60}
Artificial Intelligence: A Modern Approach, 2016, Russell \& Norvig\\
Human Rational\\
Thinking\\
Systems that think like humans\\
\end{itemize}
  \item Cognitive modelling approach
  \item Introspection, psychological  
experiments, brain imaging\\
  \item Cognitive Science
Systems that think rationally\\
  \item Laws of thought approach
  \item “Logicist” tradition
  \item Mostly rule-based
  \item Logic
Acting Systems that act like humans Systems that act rationally\\
\end{itemize}
\begin{figure}[H]
\includegraphics[width=0.5\linewidth]{page1-image-1.png}
\end{figure}
\clearpage
(Q)
Describe: 61
\clearpage
Approaches to defining AI\\
\section{61}
Artificial Intelligence: A Modern Approach, 2016, Russell \& Norvig\\
Human Rational\\
Thinking\\
Systems that think like humans\\
\begin{itemize}
  \item Cognitive modelling approach
  \item Introspection, psychological  
experiments, brain imaging\\
  \item Cognitive Science
Systems that think rationally\\
  \item Laws of thought approach
  \item “Logicist” tradition
  \item Mostly rule-based
  \item Logic
Acting\\
Systems that act like humans\\
  \item The (total) Turing Test
  \item Requires the 6 disciplines
  \item NLP, KR, Reasoning, ML,  
Computer vision, Robotics\\
Systems that act rationally\\
\end{itemize}
\begin{figure}[H]
\includegraphics[width=0.5\linewidth]{page1-image-1.png}
\end{figure}
\clearpage
(Q)
Describe: 62
\clearpage
Approaches to defining AI\\
\section{62}
Artificial Intelligence: A Modern Approach, 2016, Russell \& Norvig\\
Human Rational\\
Thinking\\
Systems that think like humans\\
\end{itemize}
  \item Cognitive modelling approach
  \item Introspection, psychological  
experiments, brain imaging\\
  \item Cognitive Science
Systems that think rationally\\
  \item Laws of thought approach
  \item “Logicist” tradition
  \item Mostly rule-based
  \item Logic
Acting\\
Systems that act like humans\\
  \item The (total) Turing Test
  \item Requires the 6 disciplines
  \item NLP, KR, Reasoning, ML,  
Computer vision, Robotics\\
Systems that act rationally\\
  \item The rational agent approach
  \item Autonomous, perceptive,  
persistent, adapts to change\\
  \item Creates and pursues goals
\end{itemize}
\begin{figure}[H]
\includegraphics[width=0.5\linewidth]{page1-image-1.png}
\end{figure}
\clearpage
(Q)
Describe: 63
\clearpage
Approaches to defining AI\\
\section{63}
Artificial Intelligence: A Modern Approach, 2016, Russell \& Norvig\\
Human Rational\\
Thinking\\
Systems that think like humans\\
\begin{itemize}
  \item Cognitive modelling approach
  \item Introspection, psychological  
experiments, brain imaging\\
  \item Cognitive Science
Systems that think rationally\\
  \item Laws of thought approach
  \item “Logicist” tradition
  \item Mostly rule-based
  \item Logic
Acting\\
Systems that act like humans\\
  \item The (total) Turing Test
  \item Requires the 6 disciplines
  \item NLP, KR, Reasoning, ML,  
Computer vision, Robotics\\
Systems that act rationally\\
  \item The rational agent approach
  \item Autonomous, perceptive,  
persistent, adapts to change\\
  \item Creates and pursues goals
\end{itemize}
\begin{figure}[H]
\includegraphics[width=0.5\linewidth]{page1-image-1.png}
\end{figure}
\clearpage
(Q)
Describe: • e.g. worms, dogs, thermostats, airplanes, robots, humans, companies, and countries.
\clearpage
An agent ‘acts’ (does something) within an environment\\
\section{• e.g. worms, dogs, thermostats, airplanes, robots, humans, companies, and countries.}
An agent acts intelligently if:\\
\end{itemize}
  \item action is appropriate for circumstances and goals
  \item flexible to changes in environment and goals
  \item learns from experience
  \item makes appropriate choices given perceptual and computational limitations
What is an Agent?\\
\end{itemize}
64\\
Artificial Intelligence: Foundations of Computational Agents, 2017, Poole \& Markworth\\
\begin{figure}[H]
\includegraphics[width=0.5\linewidth]{page1-image-1.png}
\end{figure}
\clearpage
(Q)
Describe: • An agent whose decisions and actions can be explained in terms of  computation.
\clearpage
A computational agent is:\\
\section{• An agent whose decisions and actions can be explained in terms of  computation.}
\begin{itemize}
  \item Decision can be broken down into primitive operations that can be  implemented in a 
physical device.\\
  \item Computations can take many forms
  \item The human brain (“wetware”)
  \item Computers (“hardware”)
  \item Non computational agents:
  \item wind, rain, etc.
Computational Agent\\
\end{itemize}
65\\
Artificial Intelligence: Foundations of Computational Agents, 2017, Poole \& Markworth\\
\end{itemize}
  \item \begin{figure}[H]
\includegraphics[width=0.5\linewidth]{page1-image-1.png}
\end{figure}
\clearpage
(Q)
Describe: best expected outcome.’
\clearpage
Rational agent acts to ‘achieve the best outcome or, when there is  uncertainty, the 
\section{best expected outcome.’}
  \item AI focuses on build the general principles of rational agent and components for 
constructing them\\
  \item Two key advantages of the rational-agent over others:
  \item Amenable to scientific development than approaches on human thoughts and  
behaviour\\
  \item It is more general than the “laws of thought” approach
  \item Also deals with limited rationality – acting appropriately with limited computations
Rational Agent\\
\end{itemize}
66\\
Artificial Intelligence: A Modern Approach, 2016, Russell \& Norvig\\
\begin{itemize}
  \item \begin{figure}[H]
\includegraphics[width=0.5\linewidth]{page1-image-1.png}
\end{figure}
\clearpage
(Q)
Describe: act intelligently - Poole \& Markworth
\clearpage
AI is the field that studies the synthesis and analysis of computational agents that 
\section{act intelligently - Poole \& Markworth}
  \item An agent acts intelligently if:
  \item action is appropriate for circumstances and goals
  \item flexible to changes in environment and goals
  \item learns from experience
  \item makes appropriate choices given perceptual and computational  limitations
Intelligence\\
\end{itemize}
67\\
Artificial Intelligence: Foundations of Computational Agents, 2017, Poole \& Markworth\\
\end{itemize}
  \item \begin{figure}[H]
\includegraphics[width=0.5\linewidth]{page1-image-1.png}
\end{figure}
\clearpage
(Q)
Describe: act intelligently - Poole \& Markworth
\clearpage
AI is the field that studies the synthesis and analysis of computational agents that 
\section{act intelligently - Poole \& Markworth}
  \item An agent acts intelligently if:
  \item action is appropriate for circumstances and goals
  \item flexible to changes in environment and goals
  \item learns from experience
  \item makes appropriate choices given perceptual and computational  limitations
Intelligence\\
\end{itemize}
68\\
Artificial Intelligence: Foundations of Computational Agents, 2017, Poole \& Markworth\\
\begin{figure}[H]
\includegraphics[width=0.5\linewidth]{page1-image-1.png}
\end{figure}
\clearpage
(Q)
Describe: synthesis and analysis of computational agents that act 
\clearpage
Artificial intelligence, or AI is the field that studies the \\
\section{synthesis and analysis of computational agents that act }
rationally\\
Definition of AI\\
69\\
Artificial Intelligence: A Modern Approach, 2016, Russell \& Norvig\\
Artificial Intelligence: Foundations of Computational Agents, 2017, Poole \& Markworth\\
\begin{figure}[H]
\includegraphics[width=0.5\linewidth]{page1-image-1.png}
\end{figure}
\clearpage
(Q)
Describe: 70
\clearpage
Questions?\\
\section{70}
\begin{figure}[H]
\includegraphics[width=0.5\linewidth]{page1-image-1.png}
\end{figure}
\clearpage
(Q)
Describe: • Scientific goal – understand the principles of intelligent behaviour:
\clearpage
Two types of goals: Scientific and Engineering\\
\section{• Scientific goal – understand the principles of intelligent behaviour:}
\begin{itemize}
  \item Analysis of natural and artificial agents
  \item Formulating and testing hypothesis
  \item Designing, building and experimenting with computational agents
  \item Uses a general scientific approach
  \item Focuses on building empirical systems
  \item And not on the final applications that could be deployed to use
Goals of AI\\
\end{itemize}
71\\
Artificial Intelligence: Foundations of Computational Agents, 2017, Poole \& Markworth\\
\begin{figure}[H]
\includegraphics[width=0.5\linewidth]{page1-image-1.png}
\end{figure}
\clearpage
(Q)
Describe: • Engineering goal – concerned with constructing intelligent agents
\clearpage
Two types of goals: Scientific and Engineering\\
\section{• Engineering goal – concerned with constructing intelligent agents}
\end{itemize}
  \item Focuses on the design and synthesis of useful, intelligent artefacts.
  \item Builds agents that act intelligently
  \item Agents that are useful in many real-world applications
Goals of AI\\
\end{itemize}
72\\
Artificial Intelligence: Foundations of Computational Agents, 2017, Poole \& Markworth\\
\begin{itemize}
  \item \begin{figure}[H]
\includegraphics[width=0.5\linewidth]{page1-image-1.png}
\end{figure}
\clearpage
(Q)
Describe: • Use of bots for routine, repetitive tasks
\clearpage
Workflow/Process automation
\section{• Use of bots for routine, repetitive tasks}
  \item Enhance creative tasks
  \item More time and tools to explore creative functions
  \item Increased accuracy
  \item Human errors can be reduced
  \item Better predictions \& improved decision making
  \item Predictions of risks, performance targets, tailored product offerings etc
Business Benefits of AI\\
\end{itemize}
73\\
\end{itemize}
  \item \begin{figure}[H]
\includegraphics[width=0.5\linewidth]{page1-image-1.png}
\end{figure}
\clearpage
(Q)
Describe: • There is a huge effort in mobilizing AI for health.
\clearpage
Healthcare
\section{• There is a huge effort in mobilizing AI for health.}
  \item Smart cities, transportation, security
  \item Maps, navigation systems, unmanned vehicles, route planning, security
  \item Forecasts and predictions
  \item Weather, natural disasters, earthquakes, hurricanes, stock prices, economic
  \item Agriculture
  \item Real-time data analytics help farmers to maximise their crop yields and  profits
  \item Overall lifestyle
Social benefits of AI\\
\end{itemize}
74\\
\begin{itemize}
  \item \begin{figure}[H]
\includegraphics[width=0.5\linewidth]{page1-image-1.png}
\end{figure}
\clearpage
(Q)
Describe: • Driverless cars can be hacked
\clearpage
Safety and security
\section{• Driverless cars can be hacked}
  \item Failed Facebook AI chatbot experiment
  \item Racist hijack of Microsoft AI Tweeter feed
  \item Trust and social manipulation
  \item Facebook-Cambridge Analytica Scandal
  \item Explainable (or Interpretable) AI (XAI)
  \item Deep neural models are naturally opaque
  \item Possible job losses
  \item “AI will replace more than 75 million jobs by 2022” – World Economic Forum
Risks and Challenges of AI\\
\end{itemize}
75\\
\end{itemize}
  \item \begin{figure}[H]
\includegraphics[width=0.5\linewidth]{page1-image-1.png}
\end{figure}
\clearpage
(Q)
Describe: • If AI violates ethical rules, who will be responsible?
\clearpage
Accountability
\section{• If AI violates ethical rules, who will be responsible?}
  \item Accuracy, bias, privacy and inequality
  \item AI learns from data provided by humans which may encode human biases and  
prejudices\\
  \item Facial recognition to ‘predict criminals’ sparks row over AI bias – BBC
  \item IBM abandons “biased” facial recognition tech – BBC
  \item Technological social responsibility (TSR)
  \item a conscious alignment between short- and medium-term business goals and  
longer-term societal ones – McKinsey Quarterly, August, 2019\\
Ethical Concerns of AI\\
\end{itemize}
76\\
\begin{itemize}
  \item \begin{figure}[H]
\includegraphics[width=0.5\linewidth]{page1-image-1.png}
\end{figure}
\clearpage
(Q)
Describe: • Application, history, foundations of AI
\clearpage
Artificial Intelligence: An overview
\section{• Application, history, foundations of AI}
  \item Definition of AI
  \item Goals of AI
  \item AI and the Society
  \item Benefits
  \item Risk and Challenges
  \item Ethical Issues
AI Summary\\
\end{itemize}
77\\
\begin{figure}[H]
\includegraphics[width=0.5\linewidth]{page1-image-1.png}
\end{figure}
\clearpage
(Q)
Describe: 78
\clearpage
Questions?\\
\section{78}
\begin{figure}[H]
\includegraphics[width=0.5\linewidth]{page1-image-1.png}
\end{figure}
\clearpage
(Q)
Describe: 79
\clearpage
SCC361: Artificial Intelligence\\
\section{79}
Lectures, Materials and Expectations\\
Introduction to the Module\\
Overview of Artificial Intelligence\\
Overview of Machine Learning\\
\end{itemize}
  \item \begin{figure}[H]
\includegraphics[width=0.5\linewidth]{page1-image-1.png}
\end{figure}
\clearpage
(Q)
Describe: • AI and ML, Definitions of ML, How to  
\clearpage
Overview of Machine Learning
\section{• AI and ML, Definitions of ML, How to  }
learn\\
  \item Types of Machine Learning
  \item Supervised, unsupervised, semi 
supervised\\
  \item Supervised Learning
  \item Classification and regression
  \item Unsupervised Learning
  \item Clustering and association
Introduction to Machine Learning\\
\end{itemize}
80\\
\begin{itemize}
  \item \begin{figure}[H]
\includegraphics[width=0.5\linewidth]{page1-image-1.png}
\end{figure}
\clearpage
(Q)
Describe: • i.e. depended on hand-crafted rules
\clearpage
AI systems were mostly rule-based
\section{• i.e. depended on hand-crafted rules}
  \item Machine learning drives AI
  \item Learning algorithms create a logical  
mapping from data to output\\
  \item Deep learning:
  \item a subset of ML with additional layers to  
learn deeper representations data\\
AI and Machine Learning\\
\end{itemize}
81\\
Artificial Intelligence\\
Machine Learning\\
Deep Learning\\
\begin{figure}[H]
\includegraphics[width=0.5\linewidth]{page1-image-1.png}
\end{figure}
\clearpage
(Q)
Describe: “Field of study that gives computers the ability to learn 
\clearpage
Early definition of machine learning\\
\section{“Field of study that gives computers the ability to learn }
without being explicitly programmed”\\
\begin{itemize}
  \item Arthur Samuel (1959)
\end{itemize}
  \item ML pioneer that built first “self-learning” program 
that played checkers by learning  from experience\\
  \item Inverted alpha-beta pruning widely used in decision 
tree searching\\
What is Machine Learning?\\
\end{itemize}
82\\
\begin{figure}[H]
\includegraphics[width=0.5\linewidth]{page1-image-1.png}
\end{figure}
\clearpage
(Q)
Describe: “A computer is said to learn from experience E with 
\clearpage
Another popular definition:\\
\section{“A computer is said to learn from experience E with }
respect to task T and some performance  measure P, if its \\
performance on T, as  measured by P, improved with \\
experience E”\\
\begin{itemize}
  \item Tom Mitchell (1997)
\end{itemize}
  \item Again, the key is learning from experience
  \item Not explicitly programmed
What is Machine Learning?\\
\end{itemize}
83\\
\begin{figure}[H]
\includegraphics[width=0.5\linewidth]{page1-image-1.png}
\end{figure}
\clearpage
(Q)
Describe: 84
\clearpage
What is Machine Learning?\\
\section{84}
\begin{figure}[H]
\includegraphics[width=0.5\linewidth]{page1-image-1.png}
\end{figure}
\clearpage
(Q)
Describe: “A computer is said to learn from experience E with respect to task T and some  
\clearpage
Given this definition:\\
\section{“A computer is said to learn from experience E with respect to task T and some  }
performance measure P, if its performance on T, as measured by P, improved  with \\
experience E”\\
My email program watches me mark some emails as spam, and  improves on filtering \\
spams. What is the T, E and P in the setting?\\
a. Watching me label emails as spam\\
b. Classifying emails as spam or not spam\\
c. The fraction of emails correctly classified as spam or not\\
d. None of the above – this is not a machine learning problem\\
Spam or not SPAM\\
85\\
\end{itemize}
  \item \begin{figure}[H]
\includegraphics[width=0.5\linewidth]{page1-image-1.png}
\end{figure}
\clearpage
(Q)
Describe: • Traditional Programming (Software 1.0)
\clearpage
Consider the function 𝑦𝑦 = 𝑓𝑓(𝑥𝑥) (e.g. 𝑓𝑓 𝑥𝑥 = 𝑥𝑥 )
\section{• Traditional Programming (Software 1.0)}
  \item Machine Learning (Software 2.0)
What is Machine Learning?\\
\end{itemize}
86\\
Computer\\
Data ( 𝑥𝑥 )\\
Program ( 𝑦𝑦 = 𝑥𝑥 )\\
Output ( 𝑥𝑥 )\\
Computer\\
Data ( 𝑥𝑥 )\\
Output ( 𝑥𝑥 )\\
Program ( 𝑦𝑦 = 𝑥𝑥 )\\
\begin{itemize}
  \item \begin{figure}[H]
\includegraphics[width=0.5\linewidth]{page1-image-1.png}
\end{figure}
\clearpage
(Q)
Describe: • Traditional Programming (Software 1.0)
\clearpage
Consider the function 𝑦𝑦 = 𝑓𝑓(𝑥𝑥) (e.g. 𝑓𝑓 𝑥𝑥 = 𝑥𝑥 )
\section{• Traditional Programming (Software 1.0)}
  \item Machine Learning (Software 2.0)
What is Machine Learning?\\
\end{itemize}
87\\
Computer\\
Data ( 𝑥𝑥 )\\
Program ( 𝑦𝑦 = 𝑥𝑥 )\\
Output ( 𝑥𝑥 )\\
Computer\\
Data ( 𝑥𝑥 )\\
Output ( 𝑥𝑥 )\\
Program ( 𝑦𝑦 = 𝑥𝑥 )\\
\end{itemize}
  \item \begin{figure}[H]
\includegraphics[width=0.5\linewidth]{page1-image-1.png}
\end{figure}
\clearpage
(Q)
Describe: • Accumulation of individual facts
\clearpage
Memorization
\section{• Accumulation of individual facts}
  \item Limited by
  \item Time to observe facts
  \item Memory to store facts
How things are learned\\
\end{itemize}
88\\
Declarative knowledge\\
\begin{itemize}
  \item \begin{figure}[H]
\includegraphics[width=0.5\linewidth]{page1-image-1.png}
\end{figure}
\clearpage
(Q)
Describe: • Accumulation of individual facts
\clearpage
Memorization
\section{• Accumulation of individual facts}
  \item Limited by
  \item Time to observe facts
  \item Memory to store facts
  \item Generalization
  \item Deduce new facts from old facts
  \item Limited by accuracy of deduction process
  \item Essentially a predictive activity
  \item Assumes that the past predicts the future
How things are learned\\
\end{itemize}
89\\
Imperative knowledge\\
Declarative knowledge\\
\begin{figure}[H]
\includegraphics[width=0.5\linewidth]{page1-image-1.png}
\end{figure}
\clearpage
(Q)
Describe: • Classification
\clearpage
Supervised Learning\\
\section{• Classification}
\end{itemize}
  \item Regression
Unsupervised Learning\\
  \item Clustering
  \item Association
Types of Machine Learning\\
\end{itemize}
90\\
\begin{itemize}
  \item \begin{figure}[H]
\includegraphics[width=0.5\linewidth]{page1-image-1.png}
\end{figure}
\clearpage
(Q)
Describe: particular output.
\clearpage
The algorithm learns to map an input to a 
\section{particular output.}
  \item Instances of data are presented along with their 
correctly labelled output\\
  \item Similar to a teacher-student scenario
  \item The algorithm learns from experience to predict 
new unseen data\\
  \item Two broad categories:
  \item Regression
  \item Classification
Supervised Learning\\
\end{itemize}
91\\
\begin{figure}[H]
\includegraphics[width=0.5\linewidth]{page1-image-1.png}
\end{figure}
\clearpage
(Q)
Describe: 92
\clearpage
Supervised Learning\\
\section{92}
Input labelled data Training process\\
New unseen dataAlgorithms\\
Trained Model Output\\
\end{itemize}
  \item \begin{figure}[H]
\includegraphics[width=0.5\linewidth]{page1-image-1.png}
\end{figure}
\clearpage
(Q)
Describe: • Predicts a category or a class
\clearpage
Learns from labelled data (supervised)
\section{• Predicts a category or a class}
  \item Cats|Dogs
  \item Spam|Ham
  \item Cancer|Not Cancer
  \item Attempts to separate the data into  specific 
categories (or classes or labels)\\
Classification\\
\end{itemize}
93\\
\begin{itemize}
  \item \begin{figure}[H]
\includegraphics[width=0.5\linewidth]{page1-image-1.png}
\end{figure}
\clearpage
(Q)
Describe: • Predicts a continuous-valued output
\clearpage
Learns from labelled data (supervised)
\section{• Predicts a continuous-valued output}
  \item height, price, duration etc.
  \item Consider a function 𝑦𝑦 = 𝑓𝑓 𝑥𝑥
  \item we want our model to predict 𝑦𝑦𝑖𝑖 given 𝑥𝑥𝑖𝑖
  \item 𝑥𝑥𝑖𝑖 not seen during training
  \item Typically fits some linear or quadratic curve  of 
the data plot\\
  \item Linear or logistic regression algorithms are often 
used\\
Regression\\
\end{itemize}
94\\
\end{itemize}
  \item \begin{figure}[H]
\includegraphics[width=0.5\linewidth]{page1-image-1.png}
\end{figure}
\clearpage
(Q)
Describe: • with labels e.g. spam/ham or stock price at 𝑡𝑡
\clearpage
Input data = training data
\section{• with labels e.g. spam/ham or stock price at 𝑡𝑡}
  \item In training
  \item the model makes a prediction and is corrected 
if the prediction is wrong\\
  \item Training process continues until a desired  
accuracy is achieved\\
  \item Problem types: Classification and Regression
  \item Algorithms:
  \item Logistic Regression
  \item Back Propagation Neural Network.
Supervised Learning Algorithms\\
\end{itemize}
95\\
\begin{itemize}
  \item \begin{figure}[H]
\includegraphics[width=0.5\linewidth]{page1-image-1.png}
\end{figure}
\clearpage
(Q)
Describe: 1. Predict how many students will enrol in this module in the next 3 years given the 
\clearpage
If we wish to learn models to address the following
\section{1. Predict how many students will enrol in this module in the next 3 years given the }
past enrolment data\\
2. Predict whether a student will pass the module given previous years records\\
  \item How should we proceed
a. Both are regression problems\\
b. Both are classification problems\\
c. Problem 1 is regression while Problem 2 is classification\\
d. Problem 2 is regression while Problem 1 is classification\\
Quiz: Classification vs Regression\\
\end{itemize}
96\\
\end{itemize}
  \item \begin{figure}[H]
\includegraphics[width=0.5\linewidth]{page1-image-1.png}
\end{figure}
\clearpage
(Q)
Describe: • With unsupervised learning, only the  input 
\clearpage
Remember the function 𝑦𝑦 = 𝑓𝑓 𝑥𝑥
\section{• With unsupervised learning, only the  input }
data, 𝑥𝑥, is available\\
  \item There are no corresponding labels  (classes 
or categories) i.e. no output  variable, 𝑦𝑦\\
  \item Aims at modelling the underlying  structure 
of the data\\
  \item Two main categories
  \item Clustering
  \item Association
Unsupervised Learning\\
\end{itemize}
97\\
\begin{figure}[H]
\includegraphics[width=0.5\linewidth]{page1-image-1.png}
\end{figure}
\clearpage
(Q)
Describe: 98
\clearpage
Unsupervised Learning\\
\section{98}
No labels\\
\begin{itemize}
  \item \begin{figure}[H]
\includegraphics[width=0.5\linewidth]{page1-image-1.png}
\end{figure}
\clearpage
(Q)
Describe: the inherent groupings in the  data:
\clearpage
In a clustering problem, we want to  discover 
\section{the inherent groupings in the  data:}
  \item Eg: grouping customers by purchasing  
behaviour.\\
  \item In an association rule learning problem,  we 
want to discover rules that describe  large \\
portions of your data\\
  \item E.g. people that buy 𝑿𝑿 also tend to buy 𝒀𝒀
Clustering and Association\\
\end{itemize}
99\\
\end{itemize}
  \item \begin{figure}[H]
\includegraphics[width=0.5\linewidth]{page1-image-1.png}
\end{figure}
\clearpage
(Q)
Describe: • Output not known
\clearpage
Input data in not labelled
\section{• Output not known}
  \item In training
  \item Deduces structures present in the input data
  \item Extracting general rules, reducing redundancy or  
organise data by similarity\\
  \item Problem types: clustering, dimensionality  reduction 
and association rule learning\\
  \item Algorithms:
  \item K-Means algorithm
  \item Apriori algorithm.
Unsupervised Learning Algorithms\\
\end{itemize}
10\\
0\\
\begin{itemize}
  \item \begin{figure}[H]
\includegraphics[width=0.5\linewidth]{page1-image-1.png}
\end{figure}
\clearpage
(Q)
Describe: • when we have a large amount of input data ( 𝑋𝑋 ) but 
\clearpage
Semi-supervised learning approach refers to:
\section{• when we have a large amount of input data ( 𝑋𝑋 ) but }
only some of the data is labelled ( 𝑌𝑌 )\\
  \item e.g. a photo archive where only some of the images  are 
labelled, (e.g. dog, cat, person) and the majority  are \\
unlabelled.\\
  \item Many real world problems adopt this method
  \item It can be expensive or time-consuming to label data
  \item A hybrid design often helps to bridge the gaps
  \item Algorithms:
  \item A flexible combination of supervised and unsupervised 
algorithms\\
Semi-supervised Learning\\
\end{itemize}
10\\
1\\
\begin{figure}[H]
\includegraphics[width=0.5\linewidth]{page1-image-1.png}
\end{figure}
\clearpage
(Q)
Describe: • Overview of Machine Learning
\clearpage
Today’s Lecture\\
\section{• Overview of Machine Learning}
\end{itemize}
  \item AI and ML, Definitions of ML, How to  
learn\\
  \item Types of Machine Learning
  \item Supervised, unsupervised, semi-
supervised\\
  \item Supervised Learning
  \item Classification and regression
  \item Unsupervised Learning
  \item Clustering and association
Machine Learning Summary\\
\end{itemize}
10\\
2\\
\begin{figure}[H]
\includegraphics[width=0.5\linewidth]{page1-image-1.png}
\end{figure}
\clearpage
(Q)
Describe: • Materials on Moodle / Teams Files
\clearpage
Labs: Introduction to Matlab\\
\section{• Materials on Moodle / Teams Files}
Coming up…\\
10\\
3\\
Group Day Time Room\\
SCC361/P01/01 Wednesday 11:00-13:00 FST B076\\
SCC361/P01/02 Thursday 16:00-18:00 FST B076\\
SCC361/P01/03 Friday 16:00-18:00 FST B070\\
SCC361/P01/04 Thursday 10:00-12:00 FST B070\\
SCC361/P01/05 Friday 11:00-13:00 FST B080\\
\begin{figure}[H]
\includegraphics[width=0.5\linewidth]{page1-image-1.png}
\end{figure}
\clearpage
(Q)
Describe: • Materials on Moodle / Teams Files
\clearpage
Labs: Introduction to Matlab\\
\section{• Materials on Moodle / Teams Files}
Week 2 Lectures: Features in Machine Learning and Feature Extraction\\
\begin{itemize}
  \item What are features?
  \item How to extract/represent features from text, images
Coming up…\\
\end{itemize}
10\\
4\\
Group Day Time Room\\
SCC361/P01/01 Wednesday 11:00-13:00 FST B076\\
SCC361/P01/02 Thursday 16:00-18:00 FST B076\\
SCC361/P01/03 Friday 16:00-18:00 FST B070\\
SCC361/P01/04 Thursday 10:00-12:00 FST B070\\
SCC361/P01/05 Friday 11:00-13:00 FST B080\\
\begin{figure}[H]
\includegraphics[width=0.5\linewidth]{page1-image-1.png}
\end{figure}
\clearpage
(Q)
Describe: 10
\clearpage
Questions?\\
\section{10}
5\\
\begin{figure}[H]
\includegraphics[width=0.5\linewidth]{page1-image-1.png}
\end{figure}
\clearpage
(Q)
Describe: 10
\clearpage
\clearpage
(Q)
Describe: 10
\clearpage
\\
\end{document}