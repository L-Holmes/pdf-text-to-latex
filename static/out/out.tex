\documentclass[12pt]{article}

\usepackage[T1]{fontenc}\usepackage{graphicx}\graphicspath{ {./images/} } \usepackage{float}\begin{document}
19.1\\
SC\\
C\\
 2\\
0\\
1\\
 \\
Unit 19: Advanced SQL and \\
Access Control\\
SCC201 Databases\\
Based of slides from\\
John Mariani\\
\begin{figure}[H]
\includegraphics[width=0.5\linewidth]{page1-image-1.png}
\end{figure}
\clearpage19.2\\ 
 \\
\section{In this Unit …}
\begin{figure}[H]
\includegraphics[width=0.5\linewidth]{page2-image-1.png}
\end{figure}
\clearpage19.3\\ 
 \\
\section{ADVANCED SQL : MORE ON }
SCHEMAS AND VIEWS\\
\begin{figure}[H]
\includegraphics[width=0.5\linewidth]{page3-image-1.png}
\end{figure}
\clearpage19.4\\ 
 \\
\section{Introduction}
\begin{itemize}
  \item Following on from the earlier “Schemas and 
Views” Unit.\\
  \item How to
\begin{itemize}
  \item Set up a schema in SQL
  \item Define tables in SQL
  \item Define views in SQL
\end{itemize}
  \item Notice this material focuses on full SQL
\begin{itemize}
  \item Some of these features may not be available in 
certain subset implementations of SQL\\
(such as in base MySQL or SQLite)\\
\end{itemize}
\begin{figure}[H]
\includegraphics[width=0.5\linewidth]{page4-image-1.png}
\end{figure}
\clearpage19.5\\ 
 \\
\section{Conceptual Schema Definition in SQL (1)}
\end{itemize}
  \item We will use our example from ER-to-
relational mapping:\\
DEPARTMENT DName HoD NoOfEmps\\
COURSE CName Description DName\\
STUDENT\\
FName LName RegNum BDate Address Gender DName\\
TAKES CName RegNum\\
DEPT\_LOCATIONS DName DLocation\\
\end{itemize}
\begin{figure}[H]
\includegraphics[width=0.5\linewidth]{page5-image-1.png}
\end{figure}
\clearpage19.6\\ 
 \\
\section{Conceptual Schema Definition in SQL (2)}
\end{itemize}
  \item To create schema:
\begin{itemize}
  \item CREATE SCHEMA <SCHEMA\_NAME>
AUTHORISATION <AUTH\_IDENTIFIER>\\
\begin{itemize}
  \item In our example:
\end{itemize}
  \item CREATE SCHEMA UNIVERSITY
AUTHORISATION JDOE\\
\end{itemize}
\begin{figure}[H]
\includegraphics[width=0.5\linewidth]{page6-image-1.png}
\end{figure}
\clearpage19.7\\ 
 \\
\section{Conceptual Schema Definition in SQL (3)}
\begin{itemize}
  \item To create tables for a schema explicitly:
\end{itemize}
  \item CREATE TABLE
<SCHEMA\_NAME>. <TABLE\_NAME> \\
<TABLE\_DEFINITION>\\
\begin{itemize}
  \item To create tables for a schema implicitly:
\end{itemize}
  \item CREATE TABLE
<TABLE\_NAME>\\
<TABLE\_DEFINITION>\\
  \item Schema name specified in environment is used
\end{itemize}
\begin{figure}[H]
\includegraphics[width=0.5\linewidth]{page7-image-1.png}
\end{figure}
\clearpage19.8\\ 
 \\
\section{Conceptual Schema Definition in SQL (4)}
CREATE TABLE STUDENT\\
(FNAME VARCHAR(20) NOT NULL,\\
LNAME VARCHAR(20) NOT NULL,\\
REGNUM   INT NOT NULL,\\
BDATE DATE,\\
ADDRESSVARCHAR(30) ,\\
GENDER CHAR,\\
DNAME VARCHAR(20) NOT NULL,\\
PRIMARY KEY(REGNUM),\\
FOREIGN KEY(DNAME) REFERENCES DEPARTMENT(NAME)\\
);\\
\begin{figure}[H]
\includegraphics[width=0.5\linewidth]{page8-image-1.png}
\end{figure}
\clearpage19.9\\ 
 \\
\section{Conceptual Schema Definition in SQL (5)}
CREATE TABLE DEPARTMENT\\
(NAME VARCHAR(20) NOT NULL,\\
HoD VARCHAR(20) NOT NULL,\\
NUMOFEMPS INT NOT NULL DEFAULT 1,\\
PRIMARY KEY(NAME)\\
);\\
CREATE TABLE COURSE\\
(NAME VARCHAR(20) NOT NULL,\\
DESCRIPTION VARCHAR(40),\\
DNAME VARCHAR(20) NOT NULL,\\
PRIMARY KEY(NAME),\\
FOREIGN KEY(NAME) REFERENCES DEPARTMENT(NAME)\\
);\\
\begin{figure}[H]
\includegraphics[width=0.5\linewidth]{page9-image-1.png}
\end{figure}
\clearpage19.10\\ 
 \\
\section{Conceptual Schema Definition in SQL (6)}
CREATE TABLE TAKES\\
(CNAME VARCHAR(20) NOT NULL,\\
REGNUM INT NOT NULL,\\
PRIMARY KEY(CNAME, REGNUM),\\
FOREIGN KEY(CNAME)    REFERENCES COURSE(NAME),\\
FOREIGN KEY(REGNUM) REFERENCES STUDENT(REGNUM)\\
);\\
CREATE TABLE DEPT\_LOCATIONS\\
(DNAME VARCHAR(20) NOT NULL,\\
LOCATION VARCHAR(20) NOT NULL,\\
PRIMARY KEY(DNAME, LOCATION),\\
FOREIGN KEY(DNAME) REFERENCES DEPARTMENT(NAME)\\
);\\
\begin{figure}[H]
\includegraphics[width=0.5\linewidth]{page10-image-1.png}
\end{figure}
\clearpage19.11\\ 
 \\
\section{Schema Evolution using SQL}
\begin{itemize}
  \item We can use following three commands:
\end{itemize}
  \item DROP SCHEMA
  \item DROP TABLE
  \item ALTER TABLE
\end{itemize}
\begin{figure}[H]
\includegraphics[width=0.5\linewidth]{page11-image-1.png}
\end{figure}
\clearpage19.12\\ 
 \\
\section{Views in SQL (1)}
\begin{itemize}
  \item Views are virtual tables
\end{itemize}
  \item Do not necessarily exist in physical form
  \item As opposed to base tables whose tuples are 
actually stored in a database\\
\begin{itemize}
  \item If same query frequently executed on 
database it makes sense to define view based \\
on results of query and use simpler query to \\
retrieve tuples of interest from view\\
\end{itemize}
  \item Particularly useful if original query is complex,
e.g. involves a number of joins\\
\end{itemize}
\begin{figure}[H]
\includegraphics[width=0.5\linewidth]{page12-image-1.png}
\end{figure}
\clearpage19.13\\ 
 \\
\section{Views in SQL (2)}
\begin{itemize}
  \item Use the command:
\end{itemize}
  \item CREATE VIEW <VIEW\_NAME> AS
<SQL\_QUERY>\\
  \item CREATE VIEW PHYSICS\_STUDENTS AS
SELECT *\\
FROM STUDENT\\
WHERE   DNAME = ‘Physics’ ;\\
\end{itemize}
\begin{figure}[H]
\includegraphics[width=0.5\linewidth]{page13-image-1.png}
\end{figure}
\clearpage19.14\\ 
 \\
\section{Views in SQL (3)}
\begin{itemize}
  \item Notice that virtual relations can be used in 
same way as base relations in SQL \\
statements\\
\end{itemize}
  \item SELECT FNAME, LNAME
FROM PHYSICS\_STUDENTS\\
WHERE GENDER = ‘M’\\
\begin{itemize}
  \item Views can be dropped by using command:
\end{itemize}
  \item DROP VIEW <VIEW\_NAME>
e.g.\\
DROP VIEW PHYSICS\_STUDENTS\\
\end{itemize}
\begin{figure}[H]
\includegraphics[width=0.5\linewidth]{page14-image-1.png}
\end{figure}
\clearpage19.15\\ 
 \\
\section{ACCESS CONTROL : SECURITY IN }
SQL\\
\begin{figure}[H]
\includegraphics[width=0.5\linewidth]{page15-image-1.png}
\end{figure}
\clearpage19.16\\ 
 \\
\section{Mandatory Access Control}
\begin{itemize}
  \item Each database object is assigned a certain 
classification level\\
  \item i.e. top secret, secret, confidential, unclassified
  \item The levels form a strict ordering.
  \item top secret > secret > confidential> unclassified
  \item Each subject (users or programs) is given a 
clearance level.\\
  \item To access an object, a subject requires the 
necessary clearance to read or write a database \\
object.\\
  \item See the Bell-LaPadula access control model (1974).
  \item We will not cover this approach further in this 
course.\\
\end{itemize}
\begin{figure}[H]
\includegraphics[width=0.5\linewidth]{page16-image-1.png}
\end{figure}
\clearpage19.17\\ 
 \\
\section{Discretionary Access Control}
\end{itemize}
  \item Each user is given appropriate access rights 
(or privileges) on specific database objects.\\
  \item Users obtain certain privileges when they 
create an object and can pass some or all of \\
these privileges to other users at their \\
discretion.\\
  \item This approach is used in SQL.
\end{itemize}
\begin{figure}[H]
\includegraphics[width=0.5\linewidth]{page17-image-1.png}
\end{figure}
\clearpage19.18\\ 
 \\
\section{Authorisation Identifier}
\begin{itemize}
  \item An SQL identifier used to establish the identity 
of a user.\\
  \item The DBA sets up your username and usually a 
password.\\
  \item Every SQL statement executed by the DBMS is 
performed on behalf of a specific user.\\
  \item By the access rights associated with a user, we 
can determine \\
\end{itemize}
  \item what database objects a user can reference and 
  \item what operations can be performed by that user.
\end{itemize}
\begin{figure}[H]
\includegraphics[width=0.5\linewidth]{page18-image-1.png}
\end{figure}
\clearpage19.19\\ 
 \\
\section{Ownership}
\begin{itemize}
  \item Each object created in SQL has an owner.
  \item The owner is identified by the authorisation 
identifier defined in the AUTHORIZATION \\
clause of the schema to which the object \\
belongs.\\
  \item The owner is initially the only person who 
knows that object exists and subsequently \\
perform operations on that object.\\
\end{itemize}
\begin{figure}[H]
\includegraphics[width=0.5\linewidth]{page19-image-1.png}
\end{figure}
\clearpage19.20\\ 
 \\
\section{Privileges}
\end{itemize}
  \item The ISO standard defines the following privileges, 
among others.\\
select to retrieve data from a table\\
insert to insert new rows into a table. Can be \\
restricted to specific columns.\\
update to modify rows of data in a table. Can be \\
restricted to specific columns.\\
delete to delete rows of data from a table\\
references to reference columns of a named table in \\
integrity constraints. Can be restricted to \\
specific columns.\\
\end{itemize}
\begin{figure}[H]
\includegraphics[width=0.5\linewidth]{page20-image-1.png}
\end{figure}
\clearpage19.21\\ 
 \\
\section{Create Table}
\begin{itemize}
  \item When you create a table, you are the 
owner and have full privileges.\\
  \item Other users have no access, and must be 
GRANTed permissions by the owner.\\
  \item When you create a view, you are the owner
of the view. But you may not have full \\
privileges.\\
  \item You must have select privilege on the base 
table, in order to create the view in the first \\
place.\\
\end{itemize}
\begin{figure}[H]
\includegraphics[width=0.5\linewidth]{page20-image-2.png}
\end{figure}
\clearpage19.22\\ 
 \\
\section{GRANT \{PrivilegeList | ALL PRIVILEGES\}}
ON ObjectName\\
TO \{AuthorizationList | PUBLIC\}\\
[WITH GRANT OPTION]\\
The GRANT command\\
select\\
delete\\
insert [(columnName, [...])]\\
update [(columnName, [...])]\\
references [(columnName, [...])]\\
PrivilegeList\\
\begin{figure}[H]
\includegraphics[width=0.5\linewidth]{page21-image-1.png}
\end{figure}
\clearpage19.23\\ 
 \\
\section{Examples}
GRANT ALL PRIVILEGES\\
ON Staff\\
TO Manager\\
WITH GRANT OPTION\\
The user Manager can now retrieve rows from the \\
Staff table, and also insert, update and delete.\\
The Manager can pass these privileges onto other \\
users.\\
\begin{figure}[H]
\includegraphics[width=0.5\linewidth]{page22-image-1.png}
\end{figure}
\clearpage19.24\\ 
 \\
\section{Examples}
GRANT SELECT, UPDATE (salary)\\
ON Staff\\
TO Personnel, Director\\
Gives the users Personnel and Director the privileges \\
to select and update the salary column of the Staff \\
table.\\
GRANT SELECT\\
ON Branch\\
TO PUBLIC\\
Gives all users the privilege \\
SELECT on the Branch table.\\
\begin{figure}[H]
\includegraphics[width=0.5\linewidth]{page23-image-1.png}
\end{figure}
\clearpage19.25\\ 
 \\
\section{Revoking privileges from users}
\end{itemize}
  \item The REVOKE statement can take away all or 
some of the privileges previously GRANTed.\\
REVOKE \{PrivilegeList | ALL PRIVILEGES\}\\
ON ObjectName\\
FROM \{AuthorizationList | PUBLIC\}\\
\end{itemize}
\begin{figure}[H]
\includegraphics[width=0.5\linewidth]{page24-image-1.png}
\end{figure}
\clearpage19.26\\ 
 \\
\section{Examples}
REVOKE SELECT\\
ON Branch\\
FROM PUBLIC\\
Revoke the SELECT \\
privilege on the Branch \\
table from all users.\\
REVOKE ALL PRIVILEGES\\
ON Staff\\
FROM Director\\
Revoke all privileges \\
you have given to \\
Director on the Staff \\
table.\\
\begin{figure}[H]
\includegraphics[width=0.5\linewidth]{page25-image-1.png}
\end{figure}
\clearpage19.27\\ 
 \\
\section{THE END}
\begin{figure}[H]
\includegraphics[width=0.5\linewidth]{page26-image-1.png}
\end{figure}
\clearpage\\ 
\end{document}