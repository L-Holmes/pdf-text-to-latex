\documentclass[12pt]{article}

\begin{document}
19.1
SC
C
 2
0
1
 
Unit 19: Advanced SQL and 
Access Control
SCC201 Databases
Based of slides from
John Mariani
\newpage
19.2
\section{SC}
C
 2
0
1
 
In this Unit …
\newpage
19.3
\section{SC}
C
 2
0
1
 
ADVANCED SQL : MORE ON 
SCHEMAS AND VIEWS
\newpage
19.4
\section{SC}
C
 2
0
1
 
Introduction
• Following on from the earlier “Schemas and 
Views” Unit.
• How to
– Set up a schema in SQL
– Define tables in SQL
– Define views in SQL
• Notice this material focuses on full SQL
– Some of these features may not be available in 
certain subset implementations of SQL
(such as in base MySQL or SQLite)
\newpage
19.5
\section{SC}
C
 2
0
1
 
Conceptual Schema Definition in SQL (1)
• We will use our example from ER-to-
relational mapping:
DEPARTMENT DName HoD NoOfEmps
COURSE CName Description DName
STUDENT
FName LName RegNum BDate Address Gender DName
TAKES CName RegNum
DEPT_LOCATIONS DName DLocation
\newpage
19.6
\section{SC}
C
 2
0
1
 
Conceptual Schema Definition in SQL (2)
• To create schema:
– CREATE SCHEMA <SCHEMA_NAME>
AUTHORISATION <AUTH_IDENTIFIER>
• In our example:
– CREATE SCHEMA UNIVERSITY
AUTHORISATION JDOE
\newpage
19.7
\section{SC}
C
 2
0
1
 
Conceptual Schema Definition in SQL (3)
• To create tables for a schema explicitly:
– CREATE TABLE
<SCHEMA_NAME>. <TABLE_NAME> 
<TABLE_DEFINITION>
• To create tables for a schema implicitly:
– CREATE TABLE
<TABLE_NAME>
<TABLE_DEFINITION>
– Schema name specified in environment is used
\newpage
19.8
\section{SC}
C
 2
0
1
 
Conceptual Schema Definition in SQL (4)
CREATE TABLE STUDENT
(FNAME VARCHAR(20) NOT NULL,
LNAME VARCHAR(20) NOT NULL,
REGNUM   INT NOT NULL,
BDATE DATE,
ADDRESSVARCHAR(30) ,
GENDER CHAR,
DNAME VARCHAR(20) NOT NULL,
PRIMARY KEY(REGNUM),
FOREIGN KEY(DNAME) REFERENCES DEPARTMENT(NAME)
);
\newpage
19.9
\section{SC}
C
 2
0
1
 
Conceptual Schema Definition in SQL (5)
CREATE TABLE DEPARTMENT
(NAME VARCHAR(20) NOT NULL,
HoD VARCHAR(20) NOT NULL,
NUMOFEMPS INT NOT NULL DEFAULT 1,
PRIMARY KEY(NAME)
);
CREATE TABLE COURSE
(NAME VARCHAR(20) NOT NULL,
DESCRIPTION VARCHAR(40),
DNAME VARCHAR(20) NOT NULL,
PRIMARY KEY(NAME),
FOREIGN KEY(NAME) REFERENCES DEPARTMENT(NAME)
);
\newpage
19.10
\section{SC}
C
 2
0
1
 
Conceptual Schema Definition in SQL (6)
CREATE TABLE TAKES
(CNAME VARCHAR(20) NOT NULL,
REGNUM INT NOT NULL,
PRIMARY KEY(CNAME, REGNUM),
FOREIGN KEY(CNAME)    REFERENCES COURSE(NAME),
FOREIGN KEY(REGNUM) REFERENCES STUDENT(REGNUM)
);
CREATE TABLE DEPT_LOCATIONS
(DNAME VARCHAR(20) NOT NULL,
LOCATION VARCHAR(20) NOT NULL,
PRIMARY KEY(DNAME, LOCATION),
FOREIGN KEY(DNAME) REFERENCES DEPARTMENT(NAME)
);
\newpage
19.11
\section{SC}
C
 2
0
1
 
Schema Evolution using SQL
• We can use following three commands:
– DROP SCHEMA
– DROP TABLE
– ALTER TABLE
\newpage
19.12
\section{SC}
C
 2
0
1
 
Views in SQL (1)
• Views are virtual tables
– Do not necessarily exist in physical form
– As opposed to base tables whose tuples are 
actually stored in a database
• If same query frequently executed on 
database it makes sense to define view based 
on results of query and use simpler query to 
retrieve tuples of interest from view
– Particularly useful if original query is complex,
e.g. involves a number of joins
\newpage
19.13
\section{SC}
C
 2
0
1
 
Views in SQL (2)
• Use the command:
– CREATE VIEW <VIEW_NAME> AS
<SQL_QUERY>
– CREATE VIEW PHYSICS_STUDENTS AS
SELECT *
FROM STUDENT
WHERE   DNAME = ‘Physics’ ;
\newpage
19.14
\section{SC}
C
 2
0
1
 
Views in SQL (3)
• Notice that virtual relations can be used in 
same way as base relations in SQL 
statements
– SELECT FNAME, LNAME
FROM PHYSICS_STUDENTS
WHERE GENDER = ‘M’
• Views can be dropped by using command:
– DROP VIEW <VIEW_NAME>
e.g.
DROP VIEW PHYSICS_STUDENTS
\newpage
19.15
\section{SC}
C
 2
0
1
 
ACCESS CONTROL : SECURITY IN 
SQL
\newpage
19.16
\section{SC}
C
 2
0
1
 
Mandatory Access Control
• Each database object is assigned a certain 
classification level
• i.e. top secret, secret, confidential, unclassified
• The levels form a strict ordering.
• top secret > secret > confidential> unclassified
• Each subject (users or programs) is given a 
clearance level.
• To access an object, a subject requires the 
necessary clearance to read or write a database 
object.
• See the Bell-LaPadula access control model (1974).
• We will not cover this approach further in this 
course.
\newpage
19.17
\section{SC}
C
 2
0
1
 
Discretionary Access Control
• Each user is given appropriate access rights 
(or privileges) on specific database objects.
• Users obtain certain privileges when they 
create an object and can pass some or all of 
these privileges to other users at their 
discretion.
• This approach is used in SQL.
\newpage
19.18
\section{SC}
C
 2
0
1
 
Authorisation Identifier
• An SQL identifier used to establish the identity 
of a user.
• The DBA sets up your username and usually a 
password.
• Every SQL statement executed by the DBMS is 
performed on behalf of a specific user.
• By the access rights associated with a user, we 
can determine 
– what database objects a user can reference and 
– what operations can be performed by that user.
\newpage
19.19
\section{SC}
C
 2
0
1
 
Ownership
• Each object created in SQL has an owner.
• The owner is identified by the authorisation 
identifier defined in the AUTHORIZATION 
clause of the schema to which the object 
belongs.
• The owner is initially the only person who 
knows that object exists and subsequently 
perform operations on that object.
\newpage
19.20
\section{SC}
C
 2
0
1
 
Privileges
• The ISO standard defines the following privileges, 
among others.
select to retrieve data from a table
insert to insert new rows into a table. Can be 
restricted to specific columns.
update to modify rows of data in a table. Can be 
restricted to specific columns.
delete to delete rows of data from a table
references to reference columns of a named table in 
integrity constraints. Can be restricted to 
specific columns.
\newpage
19.21
\section{SC}
C
 2
0
1
 
Create Table
• When you create a table, you are the 
owner and have full privileges.
• Other users have no access, and must be 
GRANTed permissions by the owner.
• When you create a view, you are the owner
of the view. But you may not have full 
privileges.
• You must have select privilege on the base 
table, in order to create the view in the first 
place.
\newpage
19.22
\section{SC}
C
 2
0
1
 
GRANT {PrivilegeList | ALL PRIVILEGES}
ON ObjectName
TO {AuthorizationList | PUBLIC}
[WITH GRANT OPTION]
The GRANT command
select
delete
insert [(columnName, [...])]
update [(columnName, [...])]
references [(columnName, [...])]
PrivilegeList
\newpage
19.23
\section{SC}
C
 2
0
1
 
Examples
GRANT ALL PRIVILEGES
ON Staff
TO Manager
WITH GRANT OPTION
The user Manager can now retrieve rows from the 
Staff table, and also insert, update and delete.
The Manager can pass these privileges onto other 
users.
\newpage
19.24
\section{SC}
C
 2
0
1
 
Examples
GRANT SELECT, UPDATE (salary)
ON Staff
TO Personnel, Director
Gives the users Personnel and Director the privileges 
to select and update the salary column of the Staff 
table.
GRANT SELECT
ON Branch
TO PUBLIC
Gives all users the privilege 
SELECT on the Branch table.
\newpage
19.25
\section{SC}
C
 2
0
1
 
Revoking privileges from users
• The REVOKE statement can take away all or 
some of the privileges previously GRANTed.
REVOKE {PrivilegeList | ALL PRIVILEGES}
ON ObjectName
FROM {AuthorizationList | PUBLIC}
\newpage
19.26
\section{SC}
C
 2
0
1
 
Examples
REVOKE SELECT
ON Branch
FROM PUBLIC
Revoke the SELECT 
privilege on the Branch 
table from all users.
REVOKE ALL PRIVILEGES
ON Staff
FROM Director
Revoke all privileges 
you have given to 
Director on the Staff 
table.
\newpage
19.27
\section{SC}
C
 2
0
1
 
THE END
\newpage

\end{document}